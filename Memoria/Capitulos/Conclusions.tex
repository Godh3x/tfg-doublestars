%---------------------------------------------------------------------
%
%                          Cap�tulo
%
%---------------------------------------------------------------------
% !TEX root = ../Tesis.tex

\chapter{Conclusions}
\label{en:cap9}

\begin{abstract}
In this last chapter we are going to discuss the results and analyze the reason
behind them.
\end{abstract}

As already advanced by the title of this project the system is semi-automatic
because it needs external support from a person with enough experience in the
detection of binary stars. The project was designed to, given a picture, be able
to determine with almost a 100\% accuracy that, if there was not a double star,
the sector does not contain a double star. The reason behind this decision lays
in the fact that the top priority for us is not to lose any star, even though
this will mean a higher workload at the time of reviewing the photographs in
which our application detected a possible double star.

\medskip

The results of our system are a 97\% success rejecting pictures with no binary
star present and a 33\% of the pictures containing a possible double actually
had one. Going back to our initial idea, this outcome is satisfying when it
comes to refuse pictures that do not contain this type of systems, such a high
percentage of success in the no will indeed lighten up the amount of work
required.

\medskip

As far as binary systems are concerned, the 33\% of the systems detected as
possible doubles were indeed one, this is because of a design decision. To be
precise the problem is the intensity of the stars in the recolored picture in
which colors are intensified so if two stars are very close the detector will
only notice one, \ref{en:cap9:fig:comp}. This kind of close binary systems are
pretty common so the success rate is drastically reduced.

\begin{figure}[t]
  \centering
  %
  \subfloat[][]{
    \includegraphics[width=0.3\textwidth]%
    {Imagenes/Vectorial/Conclusiones/star}
    \label{en:cap9:fig:star}
  }
  \qquad
  \subfloat[][]{
    \includegraphics[width=0.3\textwidth]%
    {Imagenes/Vectorial/Conclusiones/recolor}
    \label{en:cap9:fig:recolor}
  }
  \caption{Comparative original vs. recolored\label{en:cap9:fig:comp}}
\end{figure}

\medskip

A possible measure to prevent this kind of problem in the future, and
consequently the success rate, would be to change the centroid detection. This
new method would conduct a study of pixel intensity and density in the
picture, and use this data to decide what really is the center of a star. In
this way systems such as the one see in the example, figure
\ref{en:cap9:fig:star}, could be studied as two separate stars since between
both it is possible to appreciate an intensity variation at first sight, and it
would be seen that, in reality, they form a binary system.