%---------------------------------------------------------------------
%
%                          Cap�tulo 6
%
%---------------------------------------------------------------------
% !TEX root = ../Tesis.tex

\chapter{Estrellas dobles}

\begin{resumen}
Antes de abordar el problema que nos ocupa en este trabajo vamos a explicar
los conceptos relativos a estrellas dobles que en el se tratan.
\end{resumen}

%-------------------------------------------------------------------
\section{Centroides}
%-------------------------------------------------------------------
\label{cap7:sec:centroides}

Un punto fundamental para la detecci�n de estrellas dobles consiste en la
b�squeda y an�lisis de los centros de las estrellas, los llamados centroides.
Esto es debido a que la estrella no aparece en las im�genes como un c�rculo,
cuyo centro puede determinarse sin m�s (figura \ref{cap7:fig:cideal}), sino como una acumulaci�n de puntos
brillantes, con un contorno difuso (figura \ref{cap7:fig:creal}).

\begin{figure}[t]
  \centering
  %
  \subfloat[][]{
    \includegraphics[width=0.42\textwidth]%
    {Imagenes/Vectorial/Centroides/ideal}
    \label{cap7:fig:cideal}
  }
  \qquad
  \subfloat[][]{
    \includegraphics[width=0.42\textwidth]%
    {Imagenes/Vectorial/Centroides/real}
    \label{cap7:fig:creal}
  }
  \caption{Estrella ideal vs. Estrella real}
\end{figure}

%-------------------------------------------------------------------
\section{Coordenadas}
%-------------------------------------------------------------------
\label{cap7:sec:coordenadas}

Un sistema de coordenadas celestes no es m�s que un modo de especificar
la posici�n de un cuerpo en el espacio. Existen varios sistemas de este tipo
pero en este trabajo nos quedaremos con el sistema de coordenadas ecuatorial,
que es el usado por nuestros generadores de coordenadas.

\medskip

Antes de explicar que compone este sistema vamos a introducir dos conceptos.
En primer lugar el circulo horario de un objeto es aquel que pasa por ambos
polos e intersecta al objeto. En segundo lugar el equinoccio de primavera se da
en Marzo, exactamente en el momento en que los rayos solares golpean el ecuador
perpendicularmente. Este equinoccio es el momento cero del tiempo sideral, usado
por los astr�nomos para localizar objetos en el espacio, y por tanto el momento
cero de el sistema ecuatorial.

\medskip

En este sistema una coordenada tiene dos componentes: la ascensi�n recta, RA, y
la declinaci�n, Dec. La ascensi�n recta, figura \ref{cap7:fig:ra}, mide la
distancia angular, calculada hacia el este sobre la proyecci�n del ecuador
terrestre, entre el equinoccio de primavera y el circulo horario del objeto. La
declinaci�n mide la distancia angular de un objeto perpendicularmente respecto
de la proyecci�n del ecuador, positiva hacia el norte. Por ejemplo la proyecci�n
del polo norte tiene una declinaci�n de +90� y la del sur de -90�.

\figura{Vectorial/Coordenadas/ra}{width=.6\textwidth}{cap7:fig:ra}%
{Vista cenital de la tierra, sobre el polo norte. $\Upsilon$ representa el
equinocio de primavera.\\Fuente:
\url{https://en.wikipedia.org/wiki/File:Hour_angle_still1.png}}

\medskip

Para generar grandes cantidades de coordenadas ecuatoriales hemos creado dos
generadores de coordenadas, uno continuo y otro aleatorio. Ambos generadores
necesitan que se especifiquen los rangos en los que trabajar, que oscilan entre
los siguientes valores, ambos en grados.

\begin{itemize}
  \item RA $\in$ [0, 360].
  \item Dec $\in$ [-90, 90].
\end{itemize}

\medskip

Adem�s de estos rangos el generador aleatorio necesita el n�mero de coordenadas
deseadas, se asegurar� de que no existan repetidas, en cambio el continuo
necesita dos valores uno para el desplazamiento de la ascensi�n recta y otro
para la declinaci�n. Teniendo en cuenta que posteriormente usaremos un zoom x4
los valores que permiten solapar las fotos sin perder informaci�n son los
siguientes.

\begin{itemize}
  \item Desplazamiento RA: 0.02
  \item Desplazamiento Dec: 0.01
\end{itemize}

\medskip

Al usar el generador continuo se puede ver que la diferencia en la ascensi�n
recta no se corresponde al desplazamiento especificado, esto se debe a que
estamos trabajando en un sistema esf�rico por lo cual debemos ajustar el
desplazamiento en base a la declinaci�n.

\begin{itemize}
  \item Desplazamiento RA ajustado: Desplazamiento RA * cos(Dec)
\end{itemize}


%-------------------------------------------------------------------
\section{Atributos}
%-------------------------------------------------------------------
\label{cap7:sec:atributos}

\url{http://www.asociacionhubble.org/portal/modules/grupos/estrellasdobles/guiaobs/guiaobs.pdf}


Un sistema de estrellas binarias viene definido por una serie de par�metros que
veremos a continuaci�n. En el sistema una de las componentes es principal y la
otra secundaria, el criterio para definir cual es cual es simple, la m�s
brillante es la principal.

\medskip

La separaci�n indica la distancia entre las estrellas que conforman el sistema.

\medskip

El angulo de posici�n marca la posici�n de la secundaria respecto de la
principal.

\medskip

La magnitud es el brillo de cada estrella, cuanto m�s bajo es el valor m�s
brillante es la estrella. Como se comentaba al principio de la secci�n es este
par�metro el que determina que componente es principal.

\medskip

Y, por �ltimo, su espectro o color que esta directamente relacionado con la
temperatura de la estrella.
