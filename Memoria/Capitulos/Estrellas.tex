%---------------------------------------------------------------------
%
%                          Cap�tulo 6
%
%---------------------------------------------------------------------
% !TEX root = ../Tesis.tex

\chapter{Estrellas dobles}

\begin{resumen}
Antes de abordar el problema que nos ocupa en este trabajo vamos a explicar
los conceptos relativos a estrellas dobles que en el se tratan.
\end{resumen}

%-------------------------------------------------------------------
\section{Centroides}
%-------------------------------------------------------------------
\label{cap7:sec:centroides}

Un punto fundamental para la detecci�n de estrellas dobles consiste en la
b�squeda y an�lisis de los centros de las estrellas, los llamados centroides.
Esto es debido a que la estrella no aparece en las im�genes como un c�rculo,
cuyo centro puede determinarse sin m�s (figura \ref{cap7:fig:cideal}), sino como una acumulaci�n de puntos
brillantes, con un contorno difuso (figura \ref{cap7:fig:creal}).

\begin{figure}[t]
  \centering
  %
  \subfloat[][]{
    \includegraphics[width=0.42\textwidth]%
    {Imagenes/Vectorial/Centroides/ideal}
    \label{cap7:fig:cideal}
  }
  \qquad
  \subfloat[][]{
    \includegraphics[width=0.42\textwidth]%
    {Imagenes/Vectorial/Centroides/real}
    \label{cap7:fig:creal}
  }
  \caption{Estrella ideal vs Estrella real}
\end{figure}

%-------------------------------------------------------------------
\section{Conceptos}
%-------------------------------------------------------------------
\label{cap7:sec:conceptos}

%-------------------------------------------------------------------
\section*{\NotasBibliograficas}
%-------------------------------------------------------------------
\TocNotasBibliograficas

...
