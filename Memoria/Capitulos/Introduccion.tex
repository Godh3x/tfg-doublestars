%---------------------------------------------------------------------
%
%                          Cap�tulo 1
%
%---------------------------------------------------------------------
% !TEX root = ../Tesis.tex

\chapter{Introducci�n}
\label{cap1}

\begin{resumen}
En primer lugar vamos a aclarar qu� es una estrella doble y cu�l es su utilidad
para la comunidad cient�fica. Tambi�n plantearemos el problema que este trabajo
pretende resolver.
\end{resumen}

%-------------------------------------------------------------------
\section{Estrellas dobles}
%-------------------------------------------------------------------
\label{cap1:sec:dobles}

A lo largo de la historia los astr�nomos se han ocupado de las caracter�sticas
de los diferentes objetos que se encuentran en el espacio: estrellas, planetas,
galaxias, etc. Entre estos objetos, desde antiguo han llamado la atenci�n las
estrellas dobles.

\medskip

Un pionero en este campo fue \textbf{Willen J. Luyten} quien se dedico al
estudio de estrellas con movimiento propio, y descubrio gran cantidad de
estrellas m�ltiples sentando as� las bases para el estudio de las mismas.
\citep{luytenPM}

\medskip

Seg�n \url{http://www.astromia.com/glosario/binaria.html}

\begin{quotation}
``Una Estrella doble o Estrella binaria, es una pareja de estrellas que se
mantienen unidas por gravitaci�n y giran en torno a su centro de masas com�n.
Los periodos orbitales, que van desde minutos en el caso de dobles muy cercanas
hasta miles de a�os en el caso de parejas distantes, dependen de la separaci�n
entre las estrellas y de sus respectivas masas. La observaci�n de las �rbitas de
estrellas dobles es el �nico m�todo directo que tienen los astr�nomos para pesar
las estrellas. En el caso de parejas muy pr�ximas, su atracci�n gravitatoria
puede distorsionar la forma de las estrellas, y es posible que fluya gas de una
estrella a otra en un proceso denominado transferencia de masa.''
\end{quotation}

\medskip

Aunque la mayor�a de las estrellas que vemos son dobles o incluso m�ltiples, muy
pocas de ellas son detectables a trav�s de un telescopio debido a la gran
cercan�a de sus componentes. A estas se les llama dobles visuales, y es a las
que nos dedicamos en este trabajo.

%-------------------------------------------------------------------
\section{Problema}
%-------------------------------------------------------------------
\label{cap1:sec:problema}

El reconocimiento de estrellas binarias es un trabajo muy dedicado porque
se realiza de forma manual, es necesario analizar las fotograf�as una a una
hasta que se encuentra un par de estrellas con unas caracter�sticas similares
que encajen en el arquetipo de un sistema de estrellas dobles.

\medskip

En este trabajo nos encargaremos del marco de las estrellas con un movimiento
propio com�n elevado, es decir, parejas de estrellas que tengan un movimiento
muy r�pido, en el cual las compoenentes, direcci�n y sentido, coincidan.

%-------------------------------------------------------------------
\section{Importancia}
%-------------------------------------------------------------------
\label{cap1:sec:importancia}

Esta tipolog�a de estrellas alberga una importancia clave a la hora de
recolectar datos tales como el peso de una estrella, como se menciona en la
secci�n \ref{cap1:sec:dobles}, o en caso de ya conocer las masas su distancia
exacta.

\medskip

Otra ventaja de conocer estas estrellas radica en el descubrimiento de nubes
de materia oscura, dado que estos sistemas de estrellas se mueven en paralelo
es posible observar cambios una de las estrellas que componen el sistemas que
indiquen la presencia de cuerpos invisibles a simple vista como dichas nubes.

\citet*{dsImportancia}

%-------------------------------------------------------------------
\section{Nuestra propuesta}
%-------------------------------------------------------------------
\label{cap1:sec:propuesta}

Queremos crear un sistema de detecci�n de estrellas binarias con un alto
movimiento propio com�m y, por tanto, observables en un periodo relativamente
corto de tiempo, 50 a�os aproximadamente. Este proceso no ser�a completamente
autom�tico sino que una vez detectado una posible estrella doble tendr�a que
ser revisada manualmente para cerciorarse de que no se trata de un falso
positivo.