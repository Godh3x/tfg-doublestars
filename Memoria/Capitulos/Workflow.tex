%---------------------------------------------------------------------
%
%                          Cap�tulo 5
%
%---------------------------------------------------------------------
% !TEX root = ../Tesis.tex

\chapter{Workflow}
\label{cap5}

\begin{resumen}
Una vez resuelta la recolecci�n de datos es necesario ver que tratamientos
aplicarles para alcanzar nuestro objetivo, pero antes vamos a definir una
plataforma gen�rica sobre la cu�l desarrollar dichos procesos.
\end{resumen}

%-------------------------------------------------------------------
\section{Introducci�n}
%-------------------------------------------------------------------
\label{cap5:sec:introduccion}

En los capitulos \ref{cap3} y \ref{cap4} se describen diferentes procesos por los
que una fotograf�a ha de pasar para poder detectar sistemas de estrellas dobles,
cada una de estas etapas viene definida por un n�mero arbitrario de entradas y
salidas y un evento que al abortar la ejecuci�n permita al programa salir de
forma segura.

\medskip

Cada etapa por tanto puede ser configurada y lanzada por separado puesto que para
conectarlas solo es necesario que la salida de una etapa sea la entrada de la
siguiente, de esta forma todas las etapas pueden trabajar de forma simult�nea.
Sin embargo, creemos que esta tarea puede simplificarse mediante la creacion
de un controlador que se encarge de lanzar las etapas en diferentes hilos, para
que sigan ejecutandose a la vez, y pararlas cuando el usuario lo requiera.

%-------------------------------------------------------------------
\section{Estructura}
%-------------------------------------------------------------------
\label{cap5:sec:estructura}

La idea es generar una estructura que permita al usuario definir el orden de las
etapas de manera sencilla, y una vez definidas las lance en diferentes hilos.
Una posible representaci�n del workflow ser�a un grafo ac�clico dirigido, es
importante que sea ac�clico puesto que de otro modo la informaci�n nunca
llegar�a a un estado definitivo en el cual se acepte o rechace.

\medskip

Podemos entonces definir cada fase como una entrada en un diccionario en la cual
podemos almacenar la entrada y la salida, de este modo conectar una etapa con
otra tan solo implica igualar la entrada de una a la salida de la otra. A la
hora de lanzar el hilo es necesario sabar a que funci�n llamar para lo cual nos
encontramos con dos opciones: llamar siempre a una funcion con un nombre
determinado o simplemente incluir otro dato en el diccionario. En este caso se
ha optado por que adem�s de lo que ya guardaba se almacene la funci�n a la que
se debe llamar, esta opci�n da libertad a la hora de crear los pasos adem�s de
facilitar la posterior creaci�n de los hilos.

\medskip

Existe un �ltimo problema, algunas etapas del workflow pueden tener m�s de una
entrada o salida por lo que en el diccionario se almacenara un array de entradas
y otro de salidas, debido a esto tambi�n ser� necesario especificar el numero de
entrada o salida a la que se hace referencia al conectar los pasos.

\medskip

Una entrada del diccionario tendr�a por tanto el siguiente formato.

\begin{lstlisting}[language=Python]
flow['step'] = {
  'input': [
    flow['other']['output'][0],
    settings.directory
  ],
  'output': [
    settings.directory2
  ],
  'callback': step.run,
}
\end{lstlisting}

Adem�s se hace uso de una etapa \textit{dummy} que tan solo define la entrada
del primer paso en su salida. Una vez creadas las entradas en el diccionario
para todas las etapas solo ser� necesario ejecutar el workflow.

%-------------------------------------------------------------------
\section{Etapas}
%-------------------------------------------------------------------
\label{cap5:sec:etapas}

Si bien es cierto que el workflow esta pensado para dar libertad a la hora de
implementar las etapas tiene dos requisitos que se han de cumplir. El primero es
a la hora de definir el metodo que se utilizara como callback en el diccionario,
sus parametros deben estar ordenados de modo que reciba primero todas las
entradas, luego todas las salidas y por ultimo un evento de las clase
\textit{threading}, este orden ha de ser tenido en cuenta a la hora de crear
la entrada del diccionario descrita en la secci�n \ref{cap5:sec:estructura}. En
segundo lugar se debe asegurar que la etapa se mantendr� a la espera de nuevos
datos que procesar hasta que se active el evento que se recibe por par�metro y
que una vez recibido no se abortara la ejecuci�n del paso hasta que sea seguro.

\medskip

Para poder saber que est� pasando el workflow tambi�n monta un sistema de log,
utilizando la libreria \textit{logging} de python, al cual las etapas pueden
acceder para escribir mensajes de error o informaci�n de debug que le pueda ser
de utilidad al usuario para conocer que ha ocurrido.

\medskip

Debido a que las etapas se encuentran en continua ejecuci�n podr�a darse el caso
en el que los mismos datos se procesen varias veces, para prevenir esto las
etapas que hemos implementado mantienen un log de historia que solo modifican
ellas en el cual almacenan el identificador de los datos al acabar de
procesarlos, una de las primeras comprobaciones al recoger datos de la entrada
es comprobar si el id se encuentra en la historia del paso, de ser as� los datos
no se procesan.

%-------------------------------------------------------------------
\section{Uso}
%-------------------------------------------------------------------
\label{cap5:sec:uso}

Para utilizar el workflow son necesarios tres pasos.

\medskip

En primer lugar modificar el fichero de configuraci�n, \textit{settings.py},
en el cual se encuentran almacenados todos los directorios de los pasos que
hemos creado y los parametros de procesamiento que pueden modificarse de cada
etapa.

\medskip

Ir al workflow, \textit{workflow.py}, y editar la funci�n
\textbf{define\_flow()} para crear el diccionario de etapas tal y como se ha
explicado en la secci�n \ref{cap5:sec:estructura}.

\medskip

Por �ltimo, ejecutar el workflow. Para parar el procesamiento basta con enviar
una interrupci�n, Ctrl+C, en la consola en la cual se haya lanzado.

