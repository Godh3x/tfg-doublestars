%---------------------------------------------------------------------
%
%                          Cap�tulo 2
%
%---------------------------------------------------------------------
% !TEX root = ../Tesis.tex

\chapter{Recolecci�n de datos}
\label{cap2}

\begin{resumen}
A lo largo de este cap�tulo veremos que herramientas hemos empleado para obtener
imagenes astron�micas con las que trabajar, as� como datos de estrellas
dobles ya catalogadas.
\end{resumen}

%-------------------------------------------------------------------
\section{Aladin}
%-------------------------------------------------------------------
\label{cap2:sec:aladin}

\url{http://aladin.u-strasbg.fr/}

\medskip

\textbf{Aladin Sky Atlas} es un programa interactivo desarrollado por
el \textbf{CDS}, Centro de Datos Astron�micos de Estrasburgo, que permite
acceder recursos astron�micos digitalizados.

\medskip

Esta herramienta no solo permite visualizar imagenes de cuerpos celestes a lo
largo de los a�os, tambi�n permite realizar una superposicion de las mismas para
facilitar su comparaci�n.

%-------------------------------------------------------------------
\section{Obtenci�n de imagenes}
%-------------------------------------------------------------------
\label{cap2:sec:imgdownload}

Como se describe la secci�n \ref{cap2:sec:aladin} es posible obtener imagenes
que muestren la evolucion de los cuerpos celestes en coordenadas concretas con
a�os de diferencia.

\begin{figure}[t]
  \centering
  %
  \subfloat[][]{
    \includegraphics[width=0.42\textwidth]%
    {Imagenes/Vectorial/Obtencion/possi}
    \label{cap2:fig:star1}
  }
  \qquad
  \subfloat[][]{
    \includegraphics[width=0.42\textwidth]%
    {Imagenes/Vectorial/Obtencion/possii}
    \label{cap2:fig:star2}
  }

  \subfloat[][]{
    \includegraphics[width=0.42\textwidth]%
    {Imagenes/Vectorial/Obtencion/composicion}
    \label{cap2:fig:superpos}
  }
  \caption{Superposici�n de imagenes\label{cap2:fig:astro}}
\end{figure}

\medskip

Como se puede apreciar, figura \ref{cap2:fig:superpos}, Aladin no solo superpone
las imagenes, tambi�n las rota para tratar de encajarlas y colorea cada imagen
de un tono de modo que las estrellas pueden ser rojas o azules, dependiendo de
la imagen en que aparezcan, si el cuerpo coincide, no se ha movido, ser� blanca.

\medskip

Para facilitar el proceso de descarga aladin permite emplear scripts en
combinaci�n con ficheros de parametros. Puesto que no existe ninguna fuente que
tenga fotos de todas las coordenadas celestes son necesarios dos scripts que
emplean las siguientes fuentes.

\begin{itemize}
  \item \textbf{POSSI} y \textbf{POSSII} para descargar imagenes de coordenadas
  positivas.
  \item \textbf{POSSI} y \textbf{SERC} para las negativas.
\end{itemize}

Adem�s de ejecutar la superposici�n el script hace zoom en la coordenada
proporcionada como par�metro y almacena la imagen resultante.


%-------------------------------------------------------------------
\section{WDS}
%-------------------------------------------------------------------
\label{cap2:sec:wds}

\url{http://ad.usno.navy.mil/wds/}

\medskip

La obtenci�n de recursos de trabajo ya ha sido resuelta en la seccion
\ref{cap2:sec:imgdownload}, sin embargo, es necesario obtener un listado de
sistemas de estrellas dobles ya reconocidas que poder analizar con el fin de
semi-automatizar su reconocimiento, figura \ref{cap2:fig:wds}.

\figura{Vectorial/WDS/ProcesoWDS}{width=.5\textwidth}{cap2:fig:wds}%
{Diagrama de procesamiento para WDS}

\medskip

\textbf{WDS}, Washington Double Star Catalog, es un catalogo mantenido por
el Observatorio Naval de los Estados Unidos que recopila informaci�n sobre
sistemas de estrellas m�ltiples.

\medskip

El problema es que el catalogo almacena gran cantidad de estrellas, y nosotros
queremos casos de sistemas de estrellas dobles de movimiento r�pido. Con el fin
de obtener la informaci�n que nos interesa hemos creado un programa que filtra
dichos datos basandose en los siguientes criterios:

\begin{itemize}
\item La �ltima vez que se vi� el sistema debe ser como m�nimo 1975, esto
asegura que en los catalogos existiran almenos dos imagenes, una relativamente
actual y una de los a�os 50.
\item La magnitud de las estrellas que componen el sistema deben ser a lo
sumo 19, de este modo las estrellas seran apreciables a simple vista.
\item La separaci�n debe pertenecer al rango entre 2 y 180, las estrellas con
valores mayores est�n demasiado alejadas y desvirtuarian los resultados.
\item El desplazamiento ha de ser superior a 60 para que el movimiento sea
apreciable.
\end{itemize}

\medskip

Las coordenadas de todos los sistemas que sigan dichos criterios ser�n
almacenadas en un fichero que puede usarse como parametro de los scripts
descritos en la seccion \ref{cap2:sec:imgdownload}.
