%---------------------------------------------------------------------
%
%                          Cap�tulo 1
%
%---------------------------------------------------------------------
% !TEX root = ../Tesis.tex

\chapter{Introduction}
\label{en:cap1}

\begin{abstract}
First, we present the concept of a double star and the relevance it has
for the scientific community. Next, we state the problem this work intends
to solve.
\end{abstract}

%-------------------------------------------------------------------
\section{Double stars}
%-------------------------------------------------------------------
\label{en:cap1:sec:doubles}

Throughout history astronomers have dealt with the features of different objects
in space: stars, planets, galaxies, etc. Among them, attention has been drawn to
double stars since ancient times.

\medskip

A pioneer in this field was \textbf{Willen J. Luyten} who dedicated himself to
the study of stars with common proper motion, and discovered a large number of
multiple stars, laying the foundations for future investigation.
\citep{luytenPM}

\medskip

According to \url{http://www.astromia.com/glosario/binaria.html}.

\begin{quotation}
``A Double Star or Binary Star is a pair of stars held together by gravity
spinning around the common center of masses. The orbital periods, that oscillate
between minutes for very close doubles to millions of years for the distant
ones, depend on the separation between stars and their masses. The observation
of the orbits of double stars is the only direct method the astronomers have to
weight stars. For near pairs the gravitational pull may distort the shape of the
stars, and gas may flow from one star to another in a process called mass
transfer.''
\end{quotation}

\medskip

Even though most of the stars we see are double or even multiple, very few of
them are measurable using a telescope due to the closeness of its components.
These are called visual doubles, and that is what we are dedicated to in this
work.
\citep{newPairs}

%-------------------------------------------------------------------
\section{Issue}
%-------------------------------------------------------------------
\label{en:cap1:sec:issue}

This work focuses in the frame of double stars with high proper motion, that is,
pairs of stars with very fast movement in which the components, sense and
direction, coincide.

\medskip

The detection of binary stars is a very delicate work because it is done
manually, the pictures have to be analyzed one by one looking for a pair of
stars with similar characteristics that fit the archetype of a double star
system.

%-------------------------------------------------------------------
\section{Importance}
%-------------------------------------------------------------------
\label{en:cap1:sec:importance}

This kind of stars have a key value to collect data such as the weight of a
star, as stated in the section \ref{en:cap1:sec:doubles}, or, knowing the
masses, the exact distance.

\medskip

Another benefit of these stars lies in the discovery of dark matter clouds,
these star systems movement is parallel so it is possible to notice changes
in the expected behaviour suggesting the presence of invisible masses such as
said clouds.
\citet*{dsImportancia}

%-------------------------------------------------------------------
\section{State of the art}
%-------------------------------------------------------------------
\label{en:cap1:sec:art}

As previously stated, even though these kind of systems hold very useful data
the current detection process has not evolved much since Luyten, back then
astronomers used to overlap photographic plates from the same space sector and
study the properties of the different celestial bodies in order to check if
they are or not binary systems. Nowadays the quality of the pictures have
improved due to technological advances in the used telescopes, but it is still a
manual job that takes a lot of hours to even process the tiniest sectors
compared to the vastness of space.
\citep{luytenPhotos}

%-------------------------------------------------------------------
\section{Our approach}
%-------------------------------------------------------------------
\label{en:cap1:sec:approach}

We want to create a system capable of detecting binary stars with high proper
motion and, therefore, observable in a relatively short period of time,
approximately 50 years. This process wouldn't be completely automatic, once a
possible double star is detected it would have to be manually checked to make
sure it is not a false positive.

\medskip

In addition, we intend the filtering of images to be developed in multiple
steps and design a flow control mechanism that allows to change these steps in a
simple way.
