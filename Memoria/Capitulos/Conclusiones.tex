%---------------------------------------------------------------------
%
%                          Cap�tulo
%
%---------------------------------------------------------------------
% !TEX root = ../Tesis.tex

\chapter{Conclusiones}
\label{cap9}

\begin{resumen}
En este �ltimo cap�tulo vamos a ver los resultados obtenidos y analizar
el m�tivo de los mismos.
\end{resumen}

Como ya adelant� el t�tulo de este proyecto, el sistema es semi-autom�tico
debido a que requiere del apoyo externo de una persona con experiencia
suficiente en la detecci�n de estrellas binarias. El proyecto se dise�� de tal
manera que, dada una foto, fuera capaz de decidir con casi un 100\% de acierto
que, si no hab�a una estrella doble, dijera que no. Esto se decidi� debido a que
lo m�s importante era no perder ninguna estrella, aunque esto provocar� una
mayor carga de trabajo a la hora de revisar las fotograf�as en las que nuestra
aplicaci�n detectara una estrella doble.

\medskip

Los resultados de nuestro sistema son un 97\% de acierto en las im�genes en las
que no existe ninguna estrella binaria y un 33\% de acierto a la hora de
encontrar una estrella binaria en las que s� exista una. Volviendo a la idea
inicial que mencionamos, hemos obtenido un resultado satisfactorio cuando se
trata de rechazar fotograf�as que no contienen esta tipolog�a sitemas, un
porcentaje tan elevado de acierto reducir� en gran medida la carga de trabajo
de los expertos.

\medskip

En lo que a los sistemas binarios se refiere, el 33\% de los marcados contienen
realmente una doble esto es debido a una decisi�n inicial en el dise�o del
sistema. Concretamente a que la detecci�n se realiza con im�genes que han sido
modificadas en la fase de recoloreado. En la cual los colores se llevan a su
m�xima intensidad por lo que si dos estrellas est�n muy pr�ximas, al recolorear,
nuestro sistema las detectar� como una estrella �nica, figura
\ref{cap9:fig:comp}. Los sistemas binarios con estrellas tan cercanas son muy
comunes y, por tanto, el porcentaje de aciertos se ve dr�sticamente reducido.

\begin{figure}[t]
  \centering
  %
  \subfloat[][]{
    \includegraphics[width=0.3\textwidth]%
    {Imagenes/Vectorial/Conclusiones/star}
    \label{cap9:fig:star}
  }
  \qquad
  \subfloat[][]{
    \includegraphics[width=0.3\textwidth]%
    {Imagenes/Vectorial/Conclusiones/recolor}
    \label{cap9:fig:recolor}
  }
  \caption{Comparativa fotograf�a original y recoloreada\label{cap9:fig:comp}}
\end{figure}

\medskip

Como posible medida que se podr�a tomar para incrementar el porcentaje
de acierto se encuentra un cambio al sistema de detecci�n de centroides, de
manera que se realizase un estudio de la intensidad y densidad de los pixels
en la fotograf�a, y se determinase con estos datos que conforma un centro de
estrella. De esta manera ejemplos como el de la figura \ref{cap9:fig:star}
podr�an estudiarse como dos estrellas separadas, puesto que entre ambas es
posible apreciar a simple vista un cambio de intensidad, y se ver�a que, en
realidad, forman un sistema binario.
