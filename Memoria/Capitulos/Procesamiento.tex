%---------------------------------------------------------------------
%
%                          Cap�tulo 3
%
%---------------------------------------------------------------------
% !TEX root = ../Tesis.tex

\chapter{Procesamiento}
\label{cap3}

\begin{resumen}
A lo largo de este cap�tulo se presentaran los diferentes procesos a los que
se pueden someter los datos, entre ellos se encuentran transformaciones que ya
no se emplean en nuestro detector pero que podr�an ser �tiles si se exploran
nuevas formas de detecci�n.
\end{resumen}

%-------------------------------------------------------------------
\section{Recoloreado}
%-------------------------------------------------------------------
\label{cap3:sec:recolor}

Esta etapa emplea la biblioteca PIL, Python Imaging Library, para cargar las
im�genes y crear el canvas sobre el que guardar el resultado.

\medskip

A simple vista las im�genes obtenidas en la secci�n \ref{cap2:sec:imgdownload}
est�n coloreadas y nos permiten apreciar las diferencias f�cilmente. El problema
surge cuando queremos que el programa haga las misma extrapolaciones que hace
nuestro cerebro.

\medskip

Con el fin de facilitar la comparaci�n de los pixels la imagen pasa por un
proceso de recoloreado que la transforma pixel a pixel.

\medskip

Los colores que reconoce este proceso se definen mediante un diccionario que
almacena el nombre y composici�n del mismo. Esto facilita la modificaci�n de los
grupos reconocidos de modo que un cambio en el coloreado inicial de las im�genes
no supone un problema.

\medskip

El problema de este proceso es definir c�mo ha de interpretar la tonalidad del
pixel, para atajarlo hemos empleado la distancia Manhanttan. Se trata de
medir la distancia entre la composici�n del pixel y la del grupo, asumiendo
que ambas tonalidades son RGB y que est�n almacenadas en triplas, de la forma
(R, G, B), tan solo hemos de restar cada componente con su hom�loga y sumar los
resultados para obtener un valor. El grupo que resulte con menor valor ser�
aquel al que pertenezca el pixel.

\begin{figure}[t]
  \centering
  %
  \subfloat[][]{
    \includegraphics[width=0.42\textwidth]%
    {Imagenes/Vectorial/Recoloreado/no_recolor}
    \label{cap3:fig:norec}
  }
  \qquad
  \subfloat[][]{
    \includegraphics[width=0.42\textwidth]%
    {Imagenes/Vectorial/Recoloreado/recolor}
    \label{cap3:fig:rec}
  }
  \caption{Recoloreado usando la distancia Manhattan\label{cap3:fig:dManh}}
\end{figure}

\medskip

Tal y como muestra la imagen \ref{cap3:fig:rec} hay informaci�n que se pierde
durante el proceso debido a que las estrellas est�n muy apagadas haciendo que
la distancia al negro sea menor que al azul o rojo. Para poder atajar este
problema se realiza una comprobaci�n posterior, si el grupo asignado es negro
pero la distancia al azul o al rojo es inferior a un cierto umbral se reasigna
el grupo del pixel, imagen \ref{cap3:fig:rthresh}.

\figura{Vectorial/Recoloreado/recolor_thresh}{width=.5\textwidth}{cap3:fig:rthresh}%
{Recoloreado de la imagen \ref{cap3:fig:norec} con un umbral de 150}

\medskip

Cuando se ha obtenido el grupo del pixel se pinta del color de grupo en un
canvas, una vez se han procesado todos los pixel se almacena el resultado en un
fichero png.

%-------------------------------------------------------------------
\section{Comprobaci�n}
%-------------------------------------------------------------------
\label{cap3:sec:check}

A�n cuando el detector acepta un sistema de estrellas dobles existe un problema,
puede no tratarse de un sistema nuevo. Para solventar esto existe una
fase de comprobaci�n que se puede activar para todos los sistemas aceptados.

\medskip

Si se ha solicitado esta fase se realiza un primer paso que consiste en
conectarse a WDS y descargar los datos de los sistemas conocidos, este paso solo
se hace una vez por ejecuci�n, al comienzo, para no saturar los servidores. El
hecho de emplear los mismos datos durante una ejecuci�n completa no supone un
problema puesto que la base de datos no se altera habitualmente.

\medskip

Cuando un sistema es aceptado esta fase toma sus coordenadas y busca
coincidencias en todas las entradas descargadas, si no hay ninguna se ignora el
sistema detectado. Dejando por tanto un documento json de la siguiente forma.

\medskip

\begin{lstlisting}[language=json]
{
  "1": {
    "Angle difference": 0.5332939070674456,
    "Separation difference": 6.23413572082471,
    "Maximum separation": 117.64777940955791,
    "Separation %": 5.298982906530099,
    "PA": 151.86140052006294,
    "Separation": 28.995116847884947,
    "Proper Motion A (brightest)": [
      163.2,
      -258.4
    ],
    "Proper Motion B": [
      176.79999999999998,
      -299.2
    ]
  }
}
\end{lstlisting}

\medskip

Si por el contrario existen datos asociados al sistema se almacenan en el
archivo json junto con los datos proporcionados por el detector y el calculo del
error entre ambos, dando lugar a un archivo con el siguiente formato.

\begin{lstlisting}[language=json]
{
  "1": {
    "Angle difference": 1.419208447017608,
    "Separation difference": 3.3066248088095165,
    "Maximum separation": 123.22743201089601,
    "Separation %": 2.683351226955,
    "PA": 15.760824216099781,
    "Separation": 30.77825812822563,
    "Proper Motion A (brightest)": [
      136.0,
      0.0
    ],
    "Proper Motion B": [
      122.39999999999999,
      27.2
    ]
  },
  "wds": {
    "PA": 18.0,
    "Separation": 30.55,
    "Proper Motion A (brightest)": [
      93,
      15
    ],
    "Proper Motion B": [
      91,
      10
    ]
  },
  "error": {
    "PA": 2.239175783900219,
    "Separation": -0.22825812822562952,
    "Proper Motion A (brightest)": [
      -43.0,
      15.0
    ],
    "Proper Motion B": [
      -31.39999999999999,
      -17.2
    ]
  }
}
\end{lstlisting}

\medskip

Si bien esta fase es una primera comprobaci�n sus resultados no son definitivos
en el caso del no, puesto que solo se comprueban las coordenadas centrales de la
imagen. Si la estrella no se encuentra en el centro es imposible para este
programa determinar si el sistema ya se conoc�a o no.

%-------------------------------------------------------------------
\section{Contador de pixels}
%-------------------------------------------------------------------
\label{cap3:sec:pixel_counter}

Al igual que la etapa definida en la secci�n \ref{cap3:sec:recolor} se emplea
la biblioteca PIL. Esto no es lo �nico que comparten, el proceso al que se
someten las im�genes es muy similar.

\medskip

La idea de aplicar este proceso es obtener la composici�n de colores de cada
fotograf�a. Una vez obtenida se almacena el resultado en un fichero CSV que
se puede analizar posteriormente. La gama de colores se define mediante un
diccionario en el que se almacenan los colores puros que se quieren reconocer.

\medskip

Una vez cargada la imagen se analiza cada pixel que la compone, para conocer
el color, de entre los definidos en el diccionario, al que m�s se asemeja. como
el nombre de la etapa indica el objetivo es contar los pixels, por lo tanto
existe un contador por cada color en el cual se almacena el n�mero de pixels de
la imagen que pertenecen a dicha tonalidad.

\medskip

Al terminar de procesar es posible guardar el valor de dichos contadores o
transformarlos para obtener otros datos. En nuestro caso, decidimos almacenar
los porcentajes de pixels rojos, azules y blancos, y las proporciones tanto de
blancos como de azules con respecto de los rojos.

%-------------------------------------------------------------------
\section{Corte}
%-------------------------------------------------------------------
\label{cap3:sec:crop}

En esta etapa empleamos la biblioteca image\_slicer de Python, distribuida bajo
licencia MIT. Esta biblioteca permite dividir una fotograf�a en n piezas del
mismo tama�o.

\medskip

En este caso decidimos que las im�genes se dividieran en 9 sectores, lo cual
permite reducir el nivel de ruido en la fotograf�a pero dejando datos
suficientes en ella para posteriores an�lisis. En la figura \ref{cap3:fig:uncrop}
se puede observar que existe mucha informaci�n, sin embargo, si la dividimos en
9 fragmentos, figura \ref{cap3:fig:crop}, las im�genes resultantes tienen menos
ruido.

\medskip

Este proceso tan solo almacena el cuadrante central para reducir la carga de
memoria, debido a esto solo la imagen de la figura \ref{cap3:fig:qcenter}
aparecer�a en la salida. Si esto quisiese modificarse tan solo habr�a que
desactivar el par�metro 'only\_center' del m�todo run.

\figura{Vectorial/Corte/Original}{width=.5\textwidth}{cap3:fig:uncrop}%
{Imagen sin recortar}

\begin{figure}[t]
  \centering
  %
  \subfloat[][]{
    \includegraphics[width=0.26\textwidth]%
    {Imagenes/Vectorial/Corte/Q1}
  }
  \qquad
  \subfloat[][]{
    \includegraphics[width=0.26\textwidth]%
    {Imagenes/Vectorial/Corte/Q2}
  }
  \qquad
  \subfloat[][]{
    \includegraphics[width=0.26\textwidth]%
    {Imagenes/Vectorial/Corte/Q3}
  }

  \subfloat[][]{
    \includegraphics[width=0.26\textwidth]%
    {Imagenes/Vectorial/Corte/Q4}
  }
  \qquad
  \subfloat[][]{
    \includegraphics[width=0.26\textwidth]%
    {Imagenes/Vectorial/Corte/Q5}%
    \label{cap3:fig:qcenter}
  }
  \qquad
  \subfloat[][]{
    \includegraphics[width=0.26\textwidth]%
    {Imagenes/Vectorial/Corte/Q6}
  }

  \subfloat[][]{
    \includegraphics[width=0.26\textwidth]%
    {Imagenes/Vectorial/Corte/Q7}
  }
  \qquad
  \subfloat[][]{
    \includegraphics[width=0.26\textwidth]%
    {Imagenes/Vectorial/Corte/Q8}
  }
  \qquad
  \subfloat[][]{
    \includegraphics[width=0.26\textwidth]%
    {Imagenes/Vectorial/Corte/Q9}
  }
  \caption{Im�genes resultantes de la divisi�n\label{cap3:fig:crop}}
\end{figure}
