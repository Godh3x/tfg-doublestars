%---------------------------------------------------------------------
%
%                      resumen.tex
%
%---------------------------------------------------------------------
%
% Contiene el cap�tulo del resumen.
%
% Se crea como un cap�tulo sin numeraci�n.
%
%---------------------------------------------------------------------

\chapter{Resumen}
\cabeceraEspecial{Resumen}

La detecci�n de sistemas binarios tiene gran importancia en la astrof�sica, pues
permite, por ejemplo, calcular con precisi�n la distancia a la que se encuentran
o descubrir nubes de materia oscura mediante la detecci�n de cambios inesperados
en las �rbitas. Sin embargo, la mera ocurrencia de dos estrellas cercanas en el
cielo no supone que se trate de un sistema binario, porque pueden verse as� por
efecto de la perspectiva. Para saber con seguridad que en efecto es un sistema
binario habr�a que observar c�mo giran una alrededor de la otra, pero esto a
menudo no es posible al tratarse de �rbitas que requieren decenas de miles de
a�os. Un mecanismo indirecto para detectar nuevas estrellas binarias es
descrubir dobles con un notable movimiento propio com�n. La idea es que es
estad�sticamente improbable encontrar parejas que se muevan muy r�pido, y en la
misma direcci�n y sentido.

\medskip

Este trabajo pretende colaborar en la detecci�n de estos pares con alto
movimiento propio com�n como posbibles ``candidatas'' a estrella binarias. Para
ello utilizamos la superposici�n de im�genes tomadas por telescopios
profesionales con una diferencia temporal cercana a los 50 a�os. La detecci�n se
realiza comprobando parejas cuya posici�n var�a de forma significativa entre
ambas im�genes. Para ello hemos realizado un programa que analiza de forma
autom�tica estas im�genes y sugiere posibles candidatas que deben ser
corroboradas por el usuario. El sistema descarta el 97\% de las im�genes que no
tienen estrellas dobles, y encuentra el 33\% de las im�genes que s� la tienen.


\endinput
% Variable local para emacs, para  que encuentre el fichero maestro de
% compilaci�n y funcionen mejor algunas teclas r�pidas de AucTeX
%%%
%%% Local Variables:
%%% mode: latex
%%% TeX-master: "../Tesis.tex"
%%% End:
